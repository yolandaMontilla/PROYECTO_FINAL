\documentclass[a4paper,11pt]{article}
\usepackage[spanish]{babel}
\usepackage[utf8]{inputenc}
\usepackage{hyperref}

\begin{document}
\title{Como la crisis del Covid-19 ha afectado a la calidad del aire}
\author{Yolanda Montilla Patón}
\maketitle


\begin{abstract}
 La pandemia de la COVID-19 se ha convertido en uno de los mayores retos que la humanidad se ha tenido que enfrentar en los últimos años. Se han realizado diversos estudios para determinar si la calidad del aire puede afectar a la propagación del virus y si las medidas tomadas por los gobiernos han sido determinante en un cambio en la calidad del aire. El siguiente artículo se encuentra localizado en mi plataforma Github:\href{https://github.com/yolandaMontilla/PROYECTO_FINAL.git}{URL}.

\ En dicho artículo se va intentar mostrar como esta crisis mundial ha sido un punto de inflexión de cómo un cambio radical de nuestra rutina diaria puede suponer grandes cambios en la calidad del aire y en la propagación de enfermedades.
\textbf{\\ Palabras clave:}
Calidad del aire, Corona virus, Gases de efecto hinvernadero.

\end{abstract}


INTRODUCCIÓN
\\ \\ Los estudios científicos realizados en los primeros meses de 2020 sobre la realción entre aspectos atmosféricos y climáticos y propogación y contagio del corona virus no resultan concluyente y, en la mayoría de los casos, se trata de resultados preliminares.Wang et al.(2020b) han realizado un estudio donde analizan condiciones atmosféricas con la propagación del virus, tomando como referencia 429 ciudades de China dónde se observaron que el aumento de la humedad absoluta era una condicón desfavorable para la propagación.Wan, Tang, Feng y Lv(2020a) han elaborado un modelo de predicción de transmisión de la epidemia a partir de la temperatura y humedad relativa, según el cual aumento de 1ºC o 1\% de humedad, contribuyendo a reducir el número de reproducción efectiva de los casos. También se ha localiado una correlación con la calidad del aire , dónde ciudades con elebadores valores de partículas PM25 y dióxido de nitrógeno la pobleción es más propensa de contagiarse por el virus.
Diversos estudios todavía muy preliminares en China, Europa y Estados Unidos están relacionando la mortalidad ocasionada por la COVID-19 con la exposición a largo plazo a la contaminación atmosférica, tanto de partículas finas(PM25) como al dióxido de nitrogeno. Está relación deriva de la afección a los sistemas respiratorios e inmunitarios y eventualmente de la contribución a la transmisión del coronavirus, en la que las PM25 actuaran como vectores.
\\ Este artículo tiene como propósito, conocer la influencia que ha podido tener la climatología y la contaminación en la propagación del COVID-19 como referente el territorio Español y en busqueda de medidas de desescalamiento que no empeoren la calidad del aire.
\\ METODOLOGÍA
\\ El análisis se ha limitado en esta ocasión al NO2, por se la sustancia más directamente relacionada con el tráfico urbano. Se han recogido los datos oficiales de 129 de las 600 estaciones de medición de este contaminante existentes en España, correspondientes a las redes de las 26 principales ciudades, mayores de 150000 habitantes. la obtención de los datos se realizó a través de la página web diseñadas para publicar la información de las estaciones de control de la contaminación por las CCCAA.
\\El periodo de recopilación de la información ha comprendido entre el 1 de Marzo y el 30 de abril de 2020 y los mismos meses de los diez años anteriores(2010 a 2019), con el fin de reducir los sesgos meteorológicos y vacacinales debidos a las variaciones del teimpo y a la distribución enen cada año de los fines de semana y la Semana Santa.
\\Dentro de este periodo, que permite observar la variación a lo largo de marzo y abril de 2020 y
de ambos meses “tipo” (media de los años 2010 a 2019), se ha analizado separadamente el intervalo
entre los días 14 de marzo y 30 de abril, comparado con el correspondiente al promedio de la
década anterior, en el conjunto de las redes y en la estación orientada al tráfico más significativa
de cada ciudad, por su mayor concentración de NO2 y/o por su posición central.






\end{document}